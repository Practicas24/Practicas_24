\documentclass[12pt,a4paper]{article}
\usepackage[utf8]{inputenc}
\usepackage[spanish]{babel}

% Paquetes

\usepackage{amsmath}
\usepackage{amsfonts}
\usepackage{amssymb}
\usepackage{tikz}
\usepackage{graphicx}
\usepackage{subcaption}
\usepackage[colorlinks=true,allcolors=blue]{hyperref} % Crea las hiperreferencias
\graphicspath{ {Imagenes/} }

% Autor y titulo

\title{Memoria prácticas externa: simulacion ACTAR TPC}
\author{Daniel Vázquez Lago}

% Forma del  texto

\setlength{\parindent}{15px}
\usepackage[left=2.25cm,right=2cm,top=4cm,bottom=2cm]{geometry}

% Otros


\numberwithin{equation}{section}
\numberwithin{figure}{section}

% Comandos propios

\newcommand{\parentesis}[1]{\left( #1  \right)}
\newcommand{\parciales}[2]{\frac{\partial #1}{\partial #2}}
\newcommand{\pparciales}[2]{\parentesis{\parciales{#1}{#2}}}
\newcommand{\ccorchetes}[1]{\left[ #1  \right]}
\newcommand{\D}{\mathrm{d}}
\newcommand{\derivadas}[2]{\frac{\D #1}{\D #2}}

\newcommand{\tquad}{\quad \quad \quad}

% Comandos vectoriales

\newcommand{\pn}{\mathbf{p}}
\newcommand{\rn}{\mathbf{r}}
\newcommand{\un}{\mathbf{u}}
\newcommand{\vn}{\mathbf{v}}
\newcommand{\xn}{\mathbf{x}}

\newcommand{\Kn}{\mathbf{K}}
\newcommand{\Rn}{\mathbf{R}}
\newcommand{\Tn}{\mathbf{T}}

\begin{document}

\maketitle

\newpage

\tableofcontents

\newpage

\section{Información de las prácticas}

\section{Memoria de Actividades}

\subsection{Objetivos}

El objetivo principal de estas prácticas es reproducir la respuesta real del detector ACTAR TPC para una reacción nuclear usando el lenguaje de programación C++ y apoyándose en la herramienta ROOT diseñada por el CERN. La reacción nuclear estudiada es la siguiente:

\begin{equation}
    ^{11}\mathrm{Li}+^2\mathrm{H} \rightarrow ^3\mathrm{H}+^{10}\mathrm{Li} \Longleftrightarrow 1 + 2 \rightarrow 3 + 4
\end{equation}
El motivo por el cual esta reacción es de interes para la física nuclear moderna es que nos permitirá comprender mejor la estrucutura atómica del litio 10, un isótopo muy inestable. Conocer con rigor la estructura para isótopos inestables es fundamental en la fisica nuclear, ya que nos permitirá crear mejores modelos para las ``posiciones'' cuánticas de los protones y neutrones dentro del núcleo. \\ 

La estructura atómica podrá hallarse en función del ángulo de salida de las partículas pesadas. Consecuentemente es de vital importancia no solo tener los mejores detectores, si no poder predecir cuales cual es la mejor estrucutra para el detector (tamaño, grosor de los detectores de silicio...) de tal manera que podamos reducir la incertidumbre del factor ángulo lo máximo posible (lógicamente habrá otros factores).

\subsection{¿Qué es el ACTAR TPC?}

El ACTAR TPC (ACtive TARGet and Time Projection Chamber) es un detector gaseoso activo \footnote{un detector gaseoso activo quiere decir que el gas actúa dentro del a reacción.} de reacciones nucleares. La estructura es la siguiente: \\

% Introducir imagen de la caja

En nuestro caso vamos enviar isótopos de Litio 11 hacia el gas de deuterio hacia un detector de silicio. A medida que el isótopo vaya avanzando perderá energía (regido por la {\bf ecuación de Bethe-Bloch}), pudiendo  chocarse con un deuterio en cualquier parte. Una vez choque el litio 11 se producirá la reacción mencionada antes y otras (es posible que se produzca litio 12 de la reacción, o que no llegue a producirse la reacción). \\

En función de la reacción que ocurra la partícula se emitirá con un ángulo (incluso si se produce la misma reacción, se emitirán con ángulos diferntes, ya que puede que se exciten) y con energías diferentes. En función de la energía podremos detectar que partícula (y su excitación) ha sido emitida, mientras que el ángulo dependerá de la estrucutura atómica. Al final del detector habrá detectores de silicio, que nos permitirán medir la energía con la que llegan. Es fundamental entender que los detectores no miden todos los ángulos de salida, si no que se encuentran en una región pequeña del centro, de tal modo que las partículas ligeras (tritio, hidrógeno, deuterio) no se detectan. A la hora de hacer la simulación esto será importante, ya que no todos los litios emitidos son detectados. En función del ángulo, la energía y la distancia recorrida, puede que sean detectados o no. \\

Por otro lado dentro de la caja, donde se encuentra contenido el gas, estará un campo eléctrico que nos permitirá medir el movimiento de los átomos a través del gas, ya que este excitará los electrones del gas, que debido al campo eléctrico, se moverán hacia unos detectores formados por condensadores, de tal modo que podremos no solo detectar el recorrido, si no que podremos saber el instante exacto en el que se detectaron los electrones. De este modo, a través de la ecuación correspondiente, podremos no solo saber el punto de la colisión, si no que también el recorrido de la partícula a través del tiempo.  \\



({\color{red} mencionar la geometría de la caja, la distancia entre el campo eléctrico (pad) y el medidor, que es lo que vamos a usar para calcular la eficiencia}) \\


\subsection{Trabajo Realizado}

\subsubsection{Cálculo de la cinemática}

La primera tarea realizada fue calcular la cinemática de las partículas. Una buena aproximación de la colisión se puede hacer suponiendo una colisión elástica dentro de la relatividad especial, ya que debido a las altas energías y la baja masa de las partículas, usar la relatividad especial se hace obligatorio. Una mejor sería suponer que existe una energía perdida debido a la excitación de protones/neutrones en el núcleo de los átomos.  \\

Resolver la cinemática de la colsión consiste en calcular las energías de las diferentes partículas . El sistema de referencia de más interés es el nuestro, que llamamos laboratorio (figura \ref{Fig:2.2.01-Lab}), y es aquel en el que la partícula del litio 11 incide con una energía $T_{beam}$ sobre la partícula de deuterio que suponemos quieta. El otro sistema de referencia de interés es el sistema centro de masas (figura \ref{Fig:2.2.01-CM}). \\

Tras un cálculo podimos obtener la energía cinética de las partículas y su ángulo de salida (respecto el eje de incidencia) en el sistema laboratorio a partir de dos parámetros: el ángulo de salida de la partícula en el sistema de centro de masas, y de $T_{beam}$. Dado que este último es un valor conocido, podríamos obtener $\theta_4$ y $\theta_3$ a partir de $T_3$ y $T_4$.   \\

\begin{figure}[h!] \centering
\begin{subfigure}[b]{0.45\linewidth} \centering
    \begin{tikzpicture}[thick,scale=0.6] 
    \node (1) at (-3,0) {};
    \node (2) at (0,0) {$m_2$};
    \node (3) at (3,1) {$m_3$};
    \node (4) at (3,-1) {$m_4$};
    \node (m1) at (-1.5,0.5) {$m_1$};
    \draw[arrows={->},ultra thick] (1.east)--(2.west);
    \draw[arrows={->},ultra thick] (2.east)--(3.west);
    \draw[arrows={->},ultra thick] (2.east)--(4.west);
    \end{tikzpicture}
    \caption{Sistema Laboratorio}
    \label{Fig:2.2.01-Lab}
\end{subfigure}
\begin{subfigure}[b]{0.45\linewidth} \centering
    \begin{tikzpicture}[thick,scale=0.6] 
    \node (1) at (-3.5,0) {};
    \node (2) at (3.5,0) {};
    \node (3) at (1.8,2.8) {};
    \node (4) at (-1.8,-2.8) {};
    \draw[arrows={->},ultra thick] (1.east)--(-0.1,0)  ;
    \draw[arrows={->},ultra thick] (2.west)--(0.1,0)  ;
    \draw[arrows={->},ultra thick] (0.1,0.1)--(3.south west)  ;
    \draw[arrows={->},ultra thick] (-0.1,-0.1)--(4.north east)  ;
        
    \node (m1) at (-1.75,0.3) {$m_1$};
    \node (m2) at (1.75,0.3) {$m_2$};
    \node (m3) at (2.3,1.8) {$m_3$};
    \node (m4) at (-1.8,-1.8) {$m_4$};
    \end{tikzpicture}
    \caption{Sistema Centro de Masas}
    \label{Fig:2.2.01-CM}
\end{subfigure}
%\caption{Sistemas de referencia para las diferentes partículas}
\end{figure}

\subsubsection{Introducción a C++}

Nuestro segudno paso fue aprender la sintáxis básica de C++ y ROOT. Para esto realizamos algunos programas básicos, que consistían en realizar histogramas y algunas grásicas; así como la definición de variables y de funciones. \\

\subsubsection{Implementación de la simulación}

Una vez acabamos de introduccirnos a C++ empezamos a implementar la simulación. Lo primero que hicimos fue comprobar si los cálculos  de la cinemática realizados eran correctos. Para la implemetnación de estas ecuaciones se crearon diferentes clases, modularizando el programa principal, de tal modo que matuvimos bien organizadas lsa diferentes funciones. Dado que no conocemos el ángulo con el que van a salir las partículas, creamos un bucle for para recorrer todos los posibles ángulos. Así podemos ver que:  \\

% introducir .eps de la curva angulo-energía

Una vez comprobamos que la cinemática se calculo correctamente, podemos comenzar a construir la simulación. Esto implica implementar a la colisión los diferentes factores a tener en cuenta. Por orden fueron:

\begin{itemize}
    \item En primer lugar tenemos que reformar la manera de obtener los ángulos. Dado que en la realidad no sabemos (aunque si podamos predecir) los ángulos con los que va a salir la partícula, lo que hacemos es crear una función aleatoria para el ángulo $\theta_{CM}$ de la partícula pesada (partícula 3, litio 11). Es importante recalcar que el número de interacciones debe ser elevado. 
    

    \item Lo segundo que tenemos que hacer es ver si el litio 11 que se emite en el punto de colisión llega. Como tampoco sabemos el punto exacto en el que van a reaccionar, también haremos aleatoria esta variable. Para saber si es detectado o no, basta con extender la trayectoria de la partícula (suponiendo que se mueve en una línea recta) y ver si llega al detector. Para esto hay que tener en cuenta las dimensiones del detector. 
    

    \item También habrá que tener en cuenta las pérdidas de energía, ya que si la partícula llega con energía cero no podrá ser detectada. Para esto tenemos que implementar la ecuación de Bethe-Bloch para la partícula incidente y para la partícula pesada dentro del detector de silicio. Siempre que la partícula llegue al detector con energía cero la partícula no podrá ser detectada, por lo que tendremos que incluir este factor. 
    
    % figura de perdida de energía??    

\end{itemize}

Una vez acabamos con esto ya tendremos la simulación, en su versión más básica. Como esta es una simulación del experimento real, nosotros tendremos que volver a obtener, con los datos finales y medibles (pérdidas de energía en el silicio, ángulo de la partícula 3) los datos que queremos obtener (energía al salir de la colisión de la partícula 3, ángulo de la partícula 3). En ese caso tendremos que incluir algún tipo de incertidumbre en la medida de la pérdida de energía dentro del silicio. Estas vendrán dadas por dos factores. \\

El primero de estos es considerar que la perdída de energía es un proceso estadístico, por lo que dos partíciula con la misma energía inicial no van a depostiar la misma energía en el detector de silicio. El segundo es considerar la resolución de nuestros dectectores de energía, ya que energías muy proximas no podrán ser distinguibles. 

\begin{itemize}
    \item Para incluir el fenómeno estadístico de la pérdida de energía tendremos que tener en cuenta la energía inicial y la distancia que va a recorrer por el medio. Para incluir la energía pérdida perdida podremos calcular directamente (23/07)
    
    \item La resolución debida a medir con detectores de silicio usando semicontuctores p ha sido estudiado ampliamente por la física electrónica, por lo que simplemente usaremos datos ya tabulados para incluir. Con el grosor del detector de silicio (1.5 mm) podremos calcular la energía pérdida conociendo la energía inicial. De este modo tendremos la energía pérdida. Usando este dato como media, y $\sigma$ dada por la siguiente ecuación: 
    
    \begin{equation}
        \sigma = \frac{0.0213 \  \text{MeV}}{2.35} \sqrt{E}
    \end{equation}
    podremos obtener una $\Delta E$ aplicando una gaussiana con dicha media y esa $\sigma$ a través de la librería TRandom de ROOT. 

\end{itemize}

Una vez tenemos todo esto ya tenemos una simulación del experimento mucho mas realista, donde podamos obtener los resultados tal y como los mediríamos en el laboratorio, y compararlos con los que obtenemos en la simulación. Idealmente ambos deberán coincidir, aunque como veremos, debido a los factores anteriores (resolución, pérdidas estadísticas) no se verificará. 

% Introducir imagen

\subsection{Resultados}

Como ya hemos dicho nos interesa reducir la incertidumbre de la parte angular lo máximo posible. La incertidumbre del factor ángulo, que es lo que estamos tratando de estudiar, está relacionada con la $\sigma$ de la gaussiana que se forma para una determinada energía, por lo que en función de la energía de la partícula 4 tendremos una $\sigma$ (y una $s(\sigma)$). Sin embargo no todas los ángulos obtenidos para un rango de energías determinado se comportan de manera gaussiana, solo aquellas que se encuentran en la parte central, con ángulos grandes, tal y como podemos ver en la imagen \\

% Introducir imagen

Entonces es fundamental tener las máximas medidas posibles de angulos grandes, para así obtener una $\sigma$ con la incertidumbre lo mas pequeña posible. Para eso tenemos que hacer un estudio de la {\it eficiencia} del experimento, que es una relación calculada por ROOT a partir del número de partículas emitidas para un ángulo y el número de partículas {\it realmente} medidas. \\

A fin de maximizar la eficiencia, hemos tratado de calcular varias de estas gráficas para diferentes distancias entre el pad y el detector. Lógicamente cuanto mas cerca estén mejor se van a obtener los datos, ya que no estamos teniendo en cuenta el efecto del campo eléctrico. En un estudio mas detallado deberíamos tenerlo en cuenta, reudicendo probablemente esta eficiencia para las distancias más pequeñas. 

% Introducir imagen


\section{Conexión con el grado en física}

\section{Valoración personal}


\end{document}



